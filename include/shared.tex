% Javascript listing language
\lstdefinelanguage{JavaScript}{
  morekeywords=[1]{break, continue, delete, else, for, function, if, in,
    new, return, this, typeof, var, void, while, with},
  % Literals, primitive types, and reference types.
  morekeywords=[2]{false, null, true, boolean, number, undefined,
    Array, Boolean, Date, Math, Number, String, Object},
  % Built-ins.
  morekeywords=[3]{eval, parseInt, parseFloat, escape, unescape},
  keywordstyle=\bfseries\ttfamily,
  sensitive,
  morecomment=[s]{/*}{*/},
  morecomment=[l]//,
  morecomment=[s]{/**}{*/}, % JavaDoc style comments
  morestring=[b]',
  morestring=[b]"
}[keywords, comments, strings]
\lstdefinelanguage[ECMAScript2015]{JavaScript}[]{JavaScript}{
  morekeywords=[1]{await, async, case, catch, class, const, default, do,
    enum, export, extends, finally, from, implements, import, instanceof,
    let, static, super, switch, throw, try},
  morestring=[b]` % Interpolation strings.
}
\lstalias[]{ES6}[ECMAScript2015]{JavaScript}

% Default listing style
\lstdefinestyle{default}{%
  basicstyle=\ttfamily\Large,
  showspaces=false,
  showstringspaces=false,
  showtabs=false,
  tabsize=1
}
\lstset{style=default}

\newcommand{\codelisting}[3][]{%
  \lstinputlisting[%
    language=Python,firstline=#2,lastline=#3,#1%,
  ]{\codefile}%
}

\newcommand{\urlendpoint}[1]{127.0.0.1:5000/#1}
